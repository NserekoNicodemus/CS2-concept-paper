\documentclass{article}

\title{PARALLEL VERSION OF TARJAN'S ALGORITHM}
\author{NSEREKO NICODEMUS   15/U/22184     215023556,    }
\begin{document}

\maketitle
\section{Introduction}{In this paper is the concurrent version of algorithms based on depth-first search, all variants of Tarjan’s Algorithm namely; finding the strongly connected components (SCCs) of a graph, finding which nodes are part of the cycle in the graph, and finding which nodes are part of a “lasso”( such that there is a path from the node to a node on a cycle).}
\subsection{Background}{We basically aim at building concurrent algorithms based on depth-first search \cite{ref2} for the three problems relevant to model checking; finding its strong connected components states that are in loops, those in lassos. : If we have more than one processor at our disposal, we can solve a problem more quickly by dividing it into independent sub-problems and solving them at the same time, one on each processor.}

{We present concurrent algorithms for each of the above three problems. Our implementations typically exhibit about a four-fold speed-up over the corresponding sequential algorithms on an eight-core machine; the speed-ups are slightly better on graphs with a higher ratio of transitions to states. }

\subsection{Problem statement}{It is quite hard to import Tarjan’s algorithm in a parallel manner since according to its implementations; it uses a depth first search(DFS) which doesn’t explore already explored nodes. The strongly connected components then form subtrees of the search tree. The nodes are placed on a stack as they are visited. When the search returns from the subtree, the nodes are popped off the stack and checked to see if they are a root of a strongly connected component. If the node is a root, then it and all the nodes taken off before it form the strongly connected component.}

\subsection{Aim and Objective}
\subsubsection{General Objective }{The major goal is to provide algorithms that process concurrently for finding strongly connected components \cite{ref1} with less time consumption and this can be done by parallelizing the algorithm}
\subsubsection{Specific Objectives}
\begin{itemize}

\item	To find the strongly connected components (SCCs) of a graph (the maximal subsets S of the graph’s nodes such that for any pair of nodes there is a path from one node to another).

\item To find which nodes are part of a cycle in the graph (such that there is a non-empty path from the node to itself).

\item  To find which nodes are part of a “lasso” (such that there is a path from the node to a node on a cycle).
\end{itemize}

\subsection{Research scope}{This parallel version of tarjan’s algorithms is presented based on depth-first search, for three problems relevant to model checking: given a state graph, to find its strongly connected components, which states are in loops, and which states are in “lassos”. The algorithm typically exhibits about a four-fold speed-up over the corresponding sequential algorithms on an eight-core machine\cite{ref3}.}

\section{Methodology}{Each search is independent, and has its own control stack and Tarjan stack. A search is started at an arbitrary node startNode that has not yet been considered by any other search. Each search proceeds much as in the standard Tarjan’s Algorithm, as long as it does not encounter a node that is part of another current search. A difficulty occurs if suspending a search would create a cycle of searches, each locked on the next. Clearly we need to take some action to ensure progress. We transfer the relevant nodes of those searches into a single search, and continue, thereby removing the blocking-cycle.}

\begin{large}\textbf{Implementation}\end{large}\\
\subsection{Suspending and Resuming Searches}{Each node n includes a field blocked: List[Search], storing the searches that have encountered this node and are blocked on it. When the node is completed, those searches can be resumed. Note that testing whether n is complete and updating blocked has to be done atomically. In addition, each suspended search has a field waitingFor, storing the node it is waiting on. We record which searches are blocked on which others in a map suspended from Search to Search, encapsulated in a Suspended object. The Suspended object has an operation suspend(s: Search, n: Node) to record that s is blocked on n.}

\subsection{Scheduling}{The implementation uses a number of worker threads which execute searches. We use a Scheduler object to provide searches for workers, thereby implementing a form of task-based parallelism. The scheduler keeps track of searches that have been unblocked as a result of the blocking node becoming complete. A dormant worker can resume one of these. (Note that when a search is unblocked, the update to the Scheduler is done after the updates to the search itself, so that it is not resumed in an inconsistent state.) The algorithm can proceed in one of two different modes: rooted, where the search starts at a particular node, but the state space is not known in advance; and unrooted, where the state space is known in advance, and new searches can start at arbitrary nodes.}\cite{ref4}

\bibliography{conceptPaper}
\bibliographystyle{ieeetr}
\end{document}
